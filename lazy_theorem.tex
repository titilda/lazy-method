\documentclass[11pt,italian]{article}
\usepackage[dvipsnames]{xcolor} % more colors
\usepackage{pgf,tikz} % plotting
\usepackage{mathtools} % mathclap

\usepackage{todonotes} % notes on things to-do
\usepackage{fontspec} % better font selection
\usepackage[T1]{fontenc} % utf-8 text
\usepackage{lmodern} % fixes font
\usepackage{foreign} % foreign words typeset
\usepackage{textgreek} % greek letters in text mode
\usepackage{makecell} % in-cell newline support
\usepackage{amssymb} % math symbols in text mode
\usepackage{amsthm} % proofs and more
\usepackage{gensymb} % generic symbols
\usepackage[italian]{babel} % italian
\usepackage{ragged2e} % more consistency with ragged
\usepackage{fancyhdr} % headers
\usepackage[a4paper, total={6in, 10in}]{geometry} % paper type
\usepackage{marginnote} % margin notes
\usepackage{mparhack}
\usepackage{etoolbox} % more macros
\usepackage{bm} % bold math symbols
\usetikzlibrary{positioning,shapes,fit,calc}
\usepackage{pgfplots}
\usepackage{xfrac}
\usepackage{tabularx} % better tables
\setlength{\parindent}{0pt} % do not indent new paragraphs
\renewcommand{\epsilon}{\varepsilon} % prettier epsilon
\renewcommand{\phi}{\varphi} % prettier phi
\renewcommand\tabularxcolumn[1]{m{#1}} % for vertical centering text in X column; copied & pasted from https://tex.stackexchange.com/questions/343028/vertical-centering-of-all-columns-in-tabularx-environment/343329
\newcolumntype{Y}{>{\centering\arraybackslash}X} % Y column type for tabularx: like X, but centers
\usepackage{adjustbox} % environment that centers overflowing contnent
\usepackage[d]{esvect}
\usepackage{fancyvrb}
\renewcommand{\vec}{\vv}
\usepackage{upquote} % makes quotes & backticks work in Verbatim env
\usepackage{mathtools}% Loads amsmath
\usepackage{mdframed}

\theoremstyle{remark}
\newtheorem*{mdproposition}{Osservazione}
\newenvironment{remark}%
	{\begin{mdframed}[backgroundcolor=White]\begin{mdproposition}}%
	{\end{mdproposition}\end{mdframed}}

\title{Metodo "Lazy"\\ \large Metodo dei Fasori per Risolvere Equazioni Differenziali Lineari con Forzante Trigonometrica}
\author{Joele Andrea Ortore\\ \small Impaginato in \LaTeX\ da Alessandro Modica}
\date{2023-09-28}

\begin{document}
\maketitle

Data un'equazione differenziale lineare di ordine \(n\) con \(a_0, a_1, ..., a_n\) coefficienti reali costanti, sia la forzante \(F(t)\) una funzione trigonometrica del tipo \(\cos (\omega t)\), \(\sin (\omega t)\), o una combinazione lineare delle due:

\begin{equation}
    \sum_{i=0}^{n} a_i \cdot y^{(i)}(t) = F(t) = k_1 \cdot \cos (\omega t) + k_2 \cdot \sin (\omega t) \mbox{ con } k_1, k_2 \in \mathbb{R}
\end{equation}

è possibile individuare una soluzione particolare \(y_p\) portando l'equazione nel dominio fasoriale:

\begin{equation}
    \overline{y} \cdot \sum_{i=0}^{n} (j \omega)^i \cdot a_i = \overline{F}
\end{equation}

e risolvendola per \(\overline{y}\), ottenendo una soluzione del tipo:

\begin{equation*}
    \overline{y} = c_1 - j c_2
\end{equation*}

riportando la soluzione nel dominio del tempo, ottengo:

\begin{equation*}
    y_p(t) = c_1 \cdot \cos (\omega t) + c_2 \cdot \sin (\omega t)
\end{equation*}

che è una soluzione particolare dell'equazione.

Esplicitando la parte reale e la parte immaginaria, posso scrivere la (2) come:

\begin{equation*}
    \overline{y} \cdot \underbrace{ \left( \overbrace{\Re\left\{\sum_{i=0}^n (j \omega)^i a_i \right\}}^{\alpha} + j \overbrace{\Im\left\{\sum_{i=0}^n (j \omega)^i a_i \right\}}^{\beta} \right)}_{\mbox{chiameremo questa porzione } \gamma} = \Re\left\{\overline{F}\right\} + j \Im\left\{\overline{F}\right\}                                                                                                                                   \\
\end{equation*}


da cui:

\begin{align*}
    \overline{y} & = \frac{(\alpha - j \beta ) \left[ \Re \{\overline{F}\} + j \Im \{\overline{F}\} \right]}{|\gamma|^2}                                                                                                 \\
    \overline{y} & = \underbrace{\frac{\alpha \Re \{\overline{F}\} + \beta \Im \{\overline{F}\} }{|\gamma|^2}}_{c_1} - j \underbrace{\frac{\beta \Re \{\overline{F}\} - \alpha \Im \{\overline{F}\} }{|\gamma|^2}}_{c_2} \\
    \overline{y} & = c_1 - j c_2                                                                                                                                                                                         \\
\end{align*}

\newpage

\begin{proof}
    Vogliamo dimostrare che data un'equazione scritta come:
    \[
        \sum_{i=0}^{n} a_i \cdot y^{(i)}(t) = F(t)
    \]

    se portata nel dominio dei fasori, risolta per $\overline{y}$ e riportata tale soluzione nel dominio del tempo, si ottiene una soluzione particolare dell'equazione di partenza; procederemo quindi a ritroso dal prototipo di soluzione fornito dal Metodo di Somiglianza e ci aspetteremo dunque di trovare un risultato del tipo:

    \begin{equation}
        \sum_{i=0}^n (j \omega)^i\, \overline{y} = x_1 - j x_2
    \end{equation}

    \vspace{1em}

    Innanzitutto, la forma generale del prototipo di soluzione particolare è:

    \[
        y_p(t) = c_1 \cos (\omega t) + c_2 \sin (\omega t)
    \]

    In accordo con il Metodo di Somiglianza, la generalizzazione della forma della soluzione è:

    \begin{equation}
        c_1 \sum_{i=0} ^ n \left\{ \omega^i \underbrace{\cos \left[ \omega t + i \left( \frac{\pi}{2} \right) \right]}_{a} \right\} + c_2 \sum_{i=0} ^ n \left\{ \omega^i \underbrace{\sin \left[ \omega t + i \left( \frac{\pi}{2} \right) \right]}_{b} \right\} = y(t)
    \end{equation}

    con \(c_1, c_2\) noti, \(n\) pari al massimo ordine di derivazione presente nell'equazione differenziale.


    \begin{remark}
        Possiamo notare che i valori di \(a\) e \(b\) si ripetono ogni volta che \(i\) arriva a un multiplo di \(4\):

        \begin{align*}
            i = 0  & \implies a = \cos (\omega t + 0) = \cos (\omega t)    \\
            i = 1  & \implies a = \cos (\omega t + \frac{\pi}{2})          \\
            i = 2  & \implies a = \cos (\omega t + \pi)                    \\
            i = 3  & \implies a = \cos (\omega t + \frac{3\pi}{2})         \\
            i = 4  & \implies a = \cos (\omega t + 2\pi) = \cos (\omega t) \\
            \vdots & \quad \vdots
        \end{align*}

        Questo porta a un parallelismo interessante con i numeri complessi, infatti:

        \[ j^0 = j^4 = j^8 = \dots\ = j^{4n} = 1 \]
    \end{remark}

    Portiamo quindi la (4) nel dominio dei fasori:

    \begin{minipage}{0.65\textwidth}
        \begin{equation*}
            c_1 \sum_{i=0}^n \left\{ \omega^i e^{i\frac{\pi}{2} j} \right\} + c_2 \sum_{i=0}^n \left\{ \omega^i e^{{(i-1)}\frac{\pi}{2} j} \right\} = \overline{y}
        \end{equation*}
    \end{minipage}
    \begin{minipage}{0.35\textwidth}

        \begin{mdframed}[linecolor=CadetBlue]
            \color{CadetBlue}
            \small
            \begin{align*}
                                    &
                \sin\left( \omega t + i \frac{\pi}{2} \right) =                                                     \\
                =                   & \cos\left(\omega t + i \frac{\pi}{2} - \frac{\pi}{2}\right)                   \\
                \mbox{In fasori:\ } & e^{\left(i \frac{\pi}{2} - \frac{\pi}{2}\right) j} = e^{(i-1)\frac{\pi}{2} j} \\
            \end{align*}
        \end{mdframed}
    \end{minipage}

    \newpage

    che può essere riscritto come:

    \begin{align*}
        c_1 \sum_{i=0}^n \left\{ \omega^i j^i\right\} + c_2 \sum_{i=0}^n \left\{ \omega^i j^{i-1} \right\} & = \overline{y} \\
        c_1 \sum_{i=0}^n \left\{ \omega j \right\}^i  - j c_2 \sum_{i=0}^n \left\{ \omega j \right\}^{i}   & = \overline{y} \\
    \end{align*}

    \[
        \overline{y} = (c_1 - j c_2) \sum_{i=0}^n \left\{\omega j \right\}^i
    \]

    che equivale a:

    \[
        \overline{y} = (c_1 - j c_2) \left( \Re \left\{\sum_{i=0}^n \left[ \omega j \right]^i \right\} + j \Im \left\{ \sum_{i=0}^n \left[ \omega j \right]^i \right\} \right)
    \]

    \vspace{1em}

    \begin{align*}
        \frac
        {\overline{y}}
        { \Re \left\{\sum_{i=0}^n \left[ \omega j \right]^i \right\} + j \Im \left\{ \sum_{i=0}^n \left[ \omega j \right]^i \right\} } =
         & \ c_1 - j c_2  \\
        \\
        \frac
        {\overline{y} \cdot {\left( \Re \left\{\sum_{i=0}^n \left[ \omega j \right]^i \right\} - j \Im \left\{ \sum_{i=0}^n \left[ \omega j \right]^i \right\} \right)}}
        {\left| \Re \left\{\sum_{i=0}^n \left[ \omega j \right]^i \right\} + j \Im \left\{ \sum_{i=0}^n \left[ \omega j \right]^i \right\} \right|^2} =
         & \  c_1 - j c_2
    \end{align*}

    \begin{equation*}
        \overline{y} \cdot \left( \Re \left\{\sum_{i=0}^n \left[ \omega j \right]^i \right\} - j \Im \left\{ \sum_{i=0}^n \left[ \omega j \right]^i \right\} \right) =
        \underbrace{\left| \Re \left\{\sum_{i=0}^n \left[ \omega j \right]^i \right\} + j \Im \left\{ \sum_{i=0}^n \left[ \omega j \right]^i \right\} \right|^2}_{\mbox{Chiameremo questa porzione }P} \cdot\, (c_1 - j c_2)
    \end{equation*}

    Quindi:

    \begin{align*}
        \overline{y} \cdot \left( \Re \left\{\sum_{i=0}^n \left[ \omega j \right]^i \right\} - j \Im \left\{ \sum_{i=0}^n \left[ \omega j \right]^i \right\} \right) & = P \cdot\, (c_1 - j c_2) \\
        \overline{y} \sum_{i=0}^n \left[ \omega j \right]^i                                                                                                          & = (c_1 - j c_2) \cdot P
    \end{align*}

    Risultato coerente con quanto ci aspettavamo dalla (3).
\end{proof}

\end{document}